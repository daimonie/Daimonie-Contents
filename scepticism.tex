\abstract{This page is about scepticism, in the sense of systematic investigation of claims and beliefs. It should be applied first to the beliefs
of the individual, and then towards claims made by others. It also incorporates open mindedness, because you should not dismiss a claim
on a priori grounds. Claims should be compared to the empirical evidence presented in their favour. If this lacks, then there is no good
reason to accept the claim. This also means assessment of the evidence and its quality.}
\begin{multicols}{2}


Incredible test; look at data in fig~\ref{fig:dataserious}. Incredible test; look at data in fig~\ref{fig:dataserious2}.

\section{Scepticism}

Let me state clearly that this is my personal view of scepticism, which incorporates science and philosophy in my world view. It is distinct
from the "sceptical movement" \cite{wikism}. I do not see myself as 
part of any movement. I am not unique in that aspect. I do think that words and their meaning are important. The word does encompass the
concept I have in mind, and as such I shall use it. However, movements and ideology often lead to dogma. I do not approve of dogma. I believe
the value of the sceptic position lies also in that each member, each new generation, has built their world view from the bottom up. Each claim,
in turn, evaluated for the available evidence. In that sense, scepticism should refer the way in which an individual approaches the world, rather
than a set of alleged positions.

\section{Conspiracies}

You might have noticed that the menu button for this article is modelled after the so-called "Eye of Providence". While this symbol was originally
depicted to show the watching eye of the divine, it is now seen by conspiracy theorists as a symbol for the "New World Order" \cite{wikinwo}.
Conspiracies involve massive, secretive operations towards some illicit goal. Proponents believe that, while the massive
organisations have so far kept it hidden from the public, they have found the truth somewhere on the internet.


Conspiracy theorists are all around us, and the gish-gallop used by various of these groups often impresses non-sceptic people to whom it is presented.
There are two components I find important with regard to conspiracies. First, to spread a matter of scientific literacy so that people can find the
flaws in the gish gallop. Second, to take the most common writings apart and show how flawed the interlocking arguments really are.


To give you an example of a few conspiracies, in no particular order. There is the flat earth conspiracy, whose proponents believe the earth is flat,
apparently that classical mechanics does not work, that the governments are working together to hide this fact and finally that the entire space age
is faked. There is the GMO conspiracy, whose proponents believe that scientists and corporations are working together to maximise profits while endangering
not only the entire population but also the ecosystem. What is more, they believe that the industry owns every single scientific investigation that
ever found evidence in favour of GM safety. Another good example is the vaccine conspiracy, whose proponents believe that vaccines are inherently dangerous,
cause neurological disorders such as autism and do not serve to the benefit of the vaccinated because those diseases are not dangerous.


I hope that this gives you some idea of conspiracies. If you wonder about a specific conspiracy and how I have investigated these, you can generally
look at my blog. For instance, it rebukes flat earth gish gallop \cite{blogfe},


\section{Justification}

Why would you bother with writing things down and trying to reach others? There are several reasons. First, people end up in emotional, financial
and physical trouble by not applying a sceptical attitude. Second, conspiracies often affect other people. For instance, the vaccine conspiracy
affects the children of its proponents, not themselves. Additionally, we live in a democracy; conspiracy theorists are voters too.


Additionally, conspiracies have a tendency of spreading. The likelihood of someone believing a conspiracy is directly correlated to the number of
previous conspiracies this person believes. I believe it is important to individual people and the society in which I live that sceptical thought
underlies the positions we take. 
	

There is also a notion of interpersonal and intergenerational justice. Some might ask why I would feel a moral compulsion to affect the way
they treat themselves, their social environment and their children. Because not acting is a moral decision. Because the ethics I hold dear 
mean that I condemn domestic violence, child abuse and other things that do not directly concern me. This is not odd. Indeed, most of our
countries have already made laws concerning such issues. 


The ethical principle that makes me condemn child abuse, or makes me want our children to grow up in literacy and education, is connected to
intergenerational justice. They are the next generation, and we are responsible for them and the environment they will live in. 

\section{Effectiveness}

But isn't it wholly ineffective? Do you not just appear angry and frustrated? You might think so. Indeed, some people have told me so. Then again,
other people have told me the story of how I was effective for them. I have user statistics, which seem to indicate that people read and spend time
on my blog posts. 


Overall, I do believe that I and others are having an effect in promoting sceptical thought, scientific literacy and critical thinking. This belief
is supported by user statistics and people directly explaining that we are effective. 

\image{data_serious.jpg}{fig:dataserious}{Data thought you were serious. Then he realised.}
\image{data_serious.jpg}{fig:dataserious2}{Data2 thought you were serious. Then he realised.}
\end{multicols}