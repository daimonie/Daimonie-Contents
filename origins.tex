\abstract{This is the story of time; from the beginning to where we are now. It is also a synopsis of the scientific history of these discoveries.I have often heard the claim that science makes assumptions. That these assumptions are somehow not valid. Or even simpler, that science might be wrong; after all, they weren't there. What is the natural history, broadly? Is it possible that there was a great flood? Are scientists somehow making claims without providing enough evidence?}
  
\begin{multicols}{2}
\section{Natural History: The origins of life, the universe and everything!}


I will start with discussing the method by which science discovers the laws of Nature. Then, I will attempt to make a chronological story. That is, I will start with the big bang, towards planetary formation and onwards to life. But, for every part, I will attempt to show how things were discovered. If anything, I'd like to stress the application of the scientific method to these concepts. Specific individuals, for example what many have seen as the ‘missing link’(fig~\ref{fig:kenham}) that is running the AiG organisation, have often asserted that the scientific method is not present in these concepts. Here, I will attempt to discredit that idea.
\image{kenham.png}{fig:kenham}{And you thought gradual change couldn't affect anything. Ken Ham runs several creationist endeavors, and was involved in a debate with Bill Nye the Science Guy regarding evolution.}

\section{The scientific method}
Any evaluation of what science has discovered about this world should consider how scientists did so. A long, long time ago, the first glimpses of reason emerged in Ancient Greece. Men like Aristotle, Plato, Diocles, Pythagoras, Apollonius and Zeno. There's a reasonably long list on Wikipedia \cite{philosophers}. Now, these philosophers attempted to divine the workings of this world by reasoned thought alone. Consider, for instance, Zeno's paradox; An arrow cannot hit a turtle. Why not, you ask? Surely it does, a turtle is but slow. Ah, he said, but as the arrow moves to the space were the turtle was, the turtle has moved away! And when the arrow reaches his new place, he has moved away again! Thus, it is impossible for the arrow to hit the turtle.

In ancient Greece, you could not do the experiment to see if Zeno was right. No, you would have to use reason to show he was wrong. And so, while products of thought, such as ethical theories flourished, scientific knowledge did not. What distinguishes Science from Philosophy?

\emph{The scientific method}. Science is a quest for knowledge (Latin: Scientia). What distinguishes it from philosophy is rigorous experiment. How would science evaluate Zeno's paradox? Well, we'd probably make a dummy turtle moving away from an archer, and make the archer shoot it. Indeed, most people these days hear Zeno's paradox and think it is utter bollocks. That's because the scientific method is entwined in our society.

I'm not going to tell you the entire history; you can read that on Wikipedia\cite{historyscientificmethod} or buy a book. Precursors of the scientific method have been around for a long, long time. There's a few earlier precursors, dating back as far as 1600 BCE. Beyond that, we mostly have a list of Greek and later Arab philosophers/scientists. In my opinion, the most important person was Galileo Galilei.

It is Galileo whom broke with tradition. It is he whom used experiments as the primary research tool and presented a treatise of motion in terms of mathematics. Mathematics has a very important role in physics. Mathematics is deductive and can describe how things change. Because it is deductive, you can infer new results from previous results. It is a method by which you can \emph{predict} what things are going to do, based no previous results and theories.

And then, as often, we have to talk about Newton. Newton outlines four ‘rules of reasoning’ in his Principia Naturalis. I'm going to loosely tell you how they look. First, admit no more causes of natural things than necessary and sufficient. Here, he simply states that a cause is something that is required for something to happen. Basically, a definition. Secondly, you need no more causes than required. This is very elegant; if one hypothetical cause is sufficient to cause whatever event happens, then no more causes are needed for it to happen. Second, similar effects should then have the same causes. For instance, if rocks fall down, and rocks sink, both are probably caused by gravity.

If something seems to be true, and we can't change it, then it's probably true. This statement is a bit vague, but listen to this. We know that all things with mass fall down due to gravity. In fact, it seems we can't make anything with mass that is not affected by gravity. Therefore, it is likely that all things with mass are affected by gravity.

Finally, he formulated what is the essential strength of science. Things that have been shown to be true are to be considered true until such time where we can either make them more accurate, allow for exceptions or the like. To use gravity again; Newton's law of gravity was true until Albert Einstein came up with some new stuff. In some cases, Newton's law of gravity is still valid; in others, such as GPS systems, it isn't. Also, if something looked to be true but a new hypothesis pops up that explains the same, both can be assumed to be true. Until, again, such time where you can choose between them or combine them. As a good example, both the wave and particle descriptions of light were used; later, they were combined into wave-particle duality.

This, then, is the way of science. Assume nothing, prove everything. From that which is proven, extrapolate or combine into new hypotheses. These shall be tested. Whenever two theories are competing, that is neither is falsified, attempt to formulate a third alternative or a crucial experiment. Repeat this, ad infinitum.

There is one final distinction one must make. What is a hypothesis, what is a theory and what is a law? A hypothesis is basically an untested idea or an idea with insufficient evidence to be taken as true. Often, hypotheses are built from consistent predictions of previous results. A theory, then, is simply a hypothesis that has been shown to be true. The theory of gravity, the theory of evolution, the theory of mechanics; all these are settled issues, within certain accuracy. A law, finally, is merely a relation observed in the data. Say, hot gases expand. That's a law, of sorts. When that relation is put into formula, it is a law. It's nothing less or more than the relation seen in the data without any explanation.

The origins of spacetime: The big bang theory.
A long, long time ago, Isaac Newton proposed a theory of gravity. It was relatively simple. I'm going to show you what it looks like. However, there's no reason for you to remember it:

\begin{align}
F &= G \frac{m_1 m_2}{r^2}
\end{align}

Newton's law of gravity relates the Force exerted by one object on another, based on the distance between them. The combined `net' force on a body is equal to its change in inertia. 

To find this result, he had to do all sorts of stuff. First, he had to combine the entire field of mathematics as known at the time into something we know call Calculus. He wasn't the only one to do so; Leibniz, independently did the same thing. Based on observation, Newton also posited his three laws of motion. You've probably heard of them. An object at rest remains at rest until acted upon by a force. The rate of change of its velocity is proportional to the sum of external forces divided by inertial mass. Any action must cause an opposite reaction.

He combined these two, the mathematics and his propositions for mechanics with his proposition for a gravitational force. Together, these allowed him to derive a function describing planetary orbit. His results validated the observations Keppler had made based on a very extensive record of observations of the position these celestial bodies had in the sky. The observation was that planets moved in elleptical orbits. For a long time, these were the most accurate results we had. Note that small-scale effects of the same force were also validated.

Then, everything changed. Albert Einstein arrived on the scene, and revolutionised physics in a great many ways. A thing that marks Albert Einstein as a physicist is that he went where the evidence went. With special relativity, he basically said ‘Guys, let us consider Maxwell's results. What does that tell us?’. With gravitational theory, he did much the same. His equations are a bit harder:

\begin{align}
R_{\mu\nu} - 12 g_{\mu\nu} + g_{\mu\nu}R \Lambda &= \frac{8 \pi G}{c^4} T_{\mu\nu}
\end{align} 

If you're interested in what this equation tells you, look at my blog on this topic: \href{http://blog.daimonie.com/2016/04/general-relativity-101.html}{General Relativity 101}. The important thing to note here is that it fit in perfectly with a new type of geometry being developed around the same time. Earlier, Gauss had come to the conclusion that it's actually possible to determine if you're living on a flat surface, or on a flat one. I'm going to give you two examples. First, imagine you're drawing a line starting at the north-pole. You draw it straight down towards the equator. Then, you go to the right over the equator. After 1414 of the equator, you go straight up again, back to the north-pole. Why am I letting you think of this? Well, you just made a triangle. With three right-angles.

But, triangles have three angles that sum to 180! Indeed. But not on a curved surface. So, you can definitely determine if you're living on a spherical-object using this method. The other method is slightly harder to imagine.You can visualise it. Take an apple; draw four dots on it. Say, draw a line around the apple and draw three dots on it. Then, in the middle of the circle you've drawn, put another dot. You now have three dots on the equator and one on the north pole. Now, cut through the apple and all four dots. You can't do it! You cut a section off, and that section necessarily contains one of the four dots.Next, suppose you take a few Capital cities. Start at, say, London. Do you know the shortest distance from London to Paris? From these two to Berlin? From these, to Rome? Well, grab a large sheet of paper. Set the largest distance in there to 10 cm. Put one of the cities in the centre. Draw the 10 cm circle, then pick an arbitrary point on that circle. That point will be the largest-distance city from your origin city. From both, draw a circle with the distance to one other city as its radius. If correct, there should be a intersection between these two circles where you can put the third city. However, if that doesn't happen, don't fear. It simply means that your three cities are far apart. Here's a \href{http://www.distancefromto.net/distance-from-new-york-to-london-gb}{page} to find the distances. You can see there that the line from London to New York is drawn as a curved one; that's because a map is a flat projection.

I'll give you a hint; you can't draw that set of lines on a flat surface. A curved one, in particular, a spherical one for this specific problem, is required. Therefore, you can determine the shape, the curvature, of the surface you are on. Now, I was talking about geometry; shortly before Einstein, someone had come to the insight that one of the basic axioms - something taken for granted - in the known geometry was not necessary. Riemann was a student of Gauss, and he developed an entirely new geometry on this insight that an axiom wasn't necessary. It's called Riemann Geometry.


\image{friedmann.png}{fig:friedmann}{The possible solutions to the Friedmann equations.} 
Anyway, Einstein's gravity formula fit in perfectly with this Riemann Geometry. But this had surprising results. Most strikingly, spacetime was curved. Now, if you go onto the internet, you can find movies on this. I'm going to link one that is on \href{https://www.youtube.com/watch?v=MTY1Kje0yLg}{Youtube}. This concept might not be new for you. However, what you perhaps don't know is that there are solutions to these equations of einstein. They're called the Friedmann equations. Under the cosmological principle, which is justified for scales larger than 100 megaparsec, there are three possible solutions. Importantly, the three solutions are a closed sphere, a flat surface or an open hyperboloid. See Wikipedia's ‘shape of the universe’\cite{shape} in fig.~\ref{fig:friedmann}.

The first is a 'closed' universe. Apparently, it expands, and then contracts again. The second keeps on expanding. The third is constant. Okay, so what? Well, at first this was a mere theoretical result. Then Edwin Hubble came along. At the time, people thought that the Milky Way galaxy is the only galaxy available. He discovered that there were other galaxies around.

We're going to have to talk about physics, again. It was long known that hot matter sends out radiation - light. It's the principle of the light bulb. It's possible to measure the colours it sends out, or rather, the wavelength of the light. Several theories came forth, most based on the wave-behaviour of light. Something called the ‘ Ultraviolet catastrophe ’ was quite vital at the time. While each of these theories predicted part of the spectrum correctly, that is, the amount of light of each colour, none predicted the entire spectrum correctly. And, worse yet, all these theories predicted an infinite amount of energy 4 being sent out by hot objects. That was obviously wrong.

Planck assumed a sort of particle behaviour of light, and \emph{predicted the entire spectrum} correctly and with a finite amount of energy being sent out. This result was weird, in a way, and later led to Einstein formulating the workings of the photovoltaic effect, after which Quantum Mechanics was discovered. Anyway, what's interesting to know is that the colour (wavelength) for which the maximum amount of light is sent out is a linear function of temperature. So suppose you measure the spectrum of a star; a star is a hot object. You can determine exactly the colour for maximum light, and then derive its temperature. That's amazing.

But something weird happened. Other, independent methods of measuring the temperature of a distant star disagreed. How can that be? Well, you might recall the Doppler effect. The Doppler effect is that the sound frequency of a moving object changes; the frequency being sent out is not the one measured. Most people know it; the sound of an ambulance, for instance, displays it well. Also, police offers use it to determine the speed of vehicles and whether or not they're over the speed limit. The exact same effect also happens for light. In cosmology, it's called a red shift.

Using this, the theory predicted that all these stars for which the temperature was off were moving away from us. Using other, independent measurements, it was also possible to see how far away they were. And something peculiar was found; the farther away a star was, the faster it was moving away from us. And, strikingly, every star was moving away from us. This lead to Hubble's Law; all stars are moving away from us because space is expanding. Again, you can sort of visualise this. Put five dots on a balloon, then inflate it. Each dot is moving away from the other dots. And this Hubble's Law was also consistent with some of the Friedmann equations; an expanding universe.

And now, finally, we're getting close to the Big Bang Hypothesis. If you extrapolate back in time, an expanding universe becomes smaller, then smaller yet. From this primal state, the universe expanded. It expanded more rapid at the start, for reasons of physics. I'm sorry, but I really can't explain that part in terms most people understand. It's sort of similar to boiling water; the phase change from liquid to gas. But not exactly. Anyway, let's just grant that the universe expanded more rapidly at the start. Before inflation, the universe was incredibly hot. It was so hot, in fact, that all sorts of extremely high energy radiation was produced. Among which, of course, microwave radiation. At some point, people came up with this idea that the expansion was more rapid at the start. This predicted that microwave radiation would be generated at that time, and still be present today. If it was present today, then you would expect a constant amount of radiation from all directions. This was first predicted in 1948.

\image{bbt.png}{fig:bbt}{A visual representation of the Big Bang theory.}

In 1964, some astrophysicists were working on an unrelated topic. However, they were having some problems. They had this incredibly accurate and delicate instrument, and it had a high amount of noise. So, they tried to fix it. They re-checked their equipment, shot a few pidgeons that were nesting in the telescope, and cleaned the whole thing. And sadly, for the pidgeons, it turns out that they were not to blame. The noise was still there. So they went on to measure what the noise was, as it was clearly something. Well, the noise was constant whatever the time or direction of the telescope. Turns out, they were measuring this microwave radiation from an inflationary universe. The inflationary universe model is known as \emph{The Big Bang Hypothesis}, displayed in fig.~\ref{fig:bbt}.

Now, don't be mistaken. I'm only outlining a small bit of the most significant evidence. Furthermore, what's exciting here is that this is the scientific method at work. From things we had measured, some explanations were formulated. But, these didn't cover everything. So better explanations came up. And from there, new predictions were made. Which were found to be true. There is, at this moment, no model other than the big bang hypothesis that accounts for this evidence. None whatsoever. From the primal state forward in time, there is no other theory that accounts for all the evidence the big bang hypothesis has.

I have mentioned Hubble's law, the expansion of space, as evidence for the big bang hypothesis. I also mentioned the cosmic microwave background radiation. Is there more? The theory predicts certain ratios of helium, deuterium and lithium in the universe, which have been found. The theory allows for prediction of the shapes of galaxies that can be found; we found those. It predicts the distribution of these; again, this was found. An important one is the discovery of Primordial Gas Clouds. These are gas clouds that have been around since the big bang itself, having formed mere minutes after the big bang itself. They are unbelievably large and contain only neutral hydrogen atoms.

No, Ken Ham, "God did it" is not a theory. It's a magic skybeing that you insert everywhere. And I'm truly fine with someone believing in this creator thing. But there is no physical evidence to support the concept that the laws of nature were different in the past. Therefore, we accept that these laws of nature, as we know them, are the same in the past. If evidence is found that these laws of nature were, in fact, different in the past, then that hypothetical fact will be assimilated into our theories. But in this world, at this time, there is no such evidence. Rather, the evidence is that that the laws do remain the same.

\section{Stellar formation}
Now, we know what spacetime did and does. But one must wonder; how does that give rise to galaxies, solar systems, planets? Earlier, I mentioned that the primordial gas clouds. We're going to talk about them, a lot.

You know that mass attracts mass. We've known that since Newton first formulated his mechanics. You also know that heat increases pressure. Most of us learn the ideal gas law in science class. Or, we've held all sorts of inflated objects over a fire. Maybe we've noted that our bicycle tires are less pressurised during the winter. It won't surprise you to learn that these two can balance each other. Let's talk about this part of physics.

Pressure is force per unit area. The unit most used is Pascal, which is 1 Newton per square metre. The imperial system often resorts to pounds per square inch. Anyway, you need both a force and an area over which it is applied. How do you find that? There's a trick. Consider an union. It's a layered object. Now, think of a single, thin layer. It has mass, it has an area. And it's almost 2-dimensional, because it is thin. That's basically how we calculate it in physics. We calculate the force exerted between the thin-spherical layer and everything that lies inside it. Well, that means we have calculated a force and we know the area of a spherical shell. That's two pressures, which we can compare.

One of these, the gravitational pressure, depends on size of the cloud. The other depends on its temperature. These two are in balance; if the gravitational pressure increases, it condenses the cloud, which makes it hotter, which increases the thermodynamic pressure, and so on and so forth. Except that this is a fight gravity is better equipped to fight. It can win. And when it wins, things are going to get complicated.

First things first. The gravitational pressure increases so far that the helium atoms of the cloud are going to be pressed tightly together. It also becomes roughly spherical, especially if it wasn't rotating. If gravity is still winning the fight, it's going to break stuff. In particular, it's going to break the atoms; the electrons of the atom will free themselves and the cores remain. Well, all these cores are positively charged; they'll repel each other. But again, gravity is better equipped. The electromagnetic forces that repel are rather weak; or rather strong. A confused statement, so let me explain. The force of electromagnetism is extremely strong, but only persists for really, really short distances. On the other hand, gravity is weak, but persists over distance. In effect, only the closely placed atoms are repelling, but all the mass of the inner layers is attracting.

And now, we're getting somewhere. That cloud became spherical. Its atoms broke. And even that wasn't enough. At the core of the cloud, nuclear fusion starts; the outside gravitational pressure is so high that the core heats up to unbelievable temperatures, and it's still not enough. All that gravitational pressure is causing even the repelling force between the positively charged nuclei to loose. And thus, these cores fuse, in nuclear fusion. This generates tremendous energy. And of course, you know what I'm talking about; our gas cloud became a sun.

These early suns continued down that path. You might not be surprised that gravity can and does still win in some of these clouds. It started with stars fueled by hydrogen fusion, and then moves on to helium fusion. From helium, it moves on to carbon fusion. From carbon, it goes to metals, at which point something weird happens. For some rather complicated reasons, the energy generated in each of these processes is different. As it burns through it's carbon, it starts on metal; but these give less energy than the carbon did. At this point, the star goes supernova and basically turns into something very, very different.

\image{nebula.png}{fig:nebula}{The nebula PIA 04934. It is called the  Elephant's Trunk nebula.}

All these processes are intriguing, really. If this interests you, you should really find someone far more eloquent to tell you. My story about stars ends there, because we need to get working on planets. All the earlier fusion processes also form things they don't burn. As soon as a new element becomes available, it too can be used in fusion. And so, all the elements we know are created. As soon as carbon fusion is going on, for instance, it's also making metal. In far lesser amounts, yes, but it is doing that as well. Well, some fraction of that metal might also fuse, to a heavier metal. And so on and so forth. When the star goes supernova, throwing off all the outer shells, it spreads an extremely large interstellar cloud of leftover fuel, of all these elements it created. 5 The leftover is one of several types of Nebulae. I put a pretty one to the left. However, you should really just go to a image-search machine and enter Nebula. It is amazing - see, for instance, the Elephant's Trunk nebula in fig.~\ref{fig:nebula}.

Suppose we have that type of nebula; A large cloud of leftover hydrogen and various kinds of elements. Suppose, also, that it is rotating around some axis as a result over the explosion. Well, the rotating force is a centrifuge; it throws out the heavier stuff, while the light stuff is in the middle. And the process starts all over again. The light stuff in the middle is hydrogen; it becomes a new star. As it contracts, it can't stop rotating. So it turns into a disc; it is called the protoplanetary disc. The heavy stuff on the outside of the disc also going to pull together under gravity. This is called accretion. The dust, as it is referred to, becomes larger and larger grains of heavy material. And these grains, when they happen to come together, stick. The debris becomes larger and larger, until at some point they are planet and moon sized.

Such a nice story. But, to be fair, we should be talking about evidence as well. The idea is generally to use our awesome telescopes and look around. We can find all sorts of objects when we look around. Don't forget - light takes time to travel. Some objects are so far away that it can take millions of years for light to travel to us; as a result, we can still see the formation of all sorts of things around. We can analyse that light, see what is happening. What elements are involved, what processes are around.

Sadly, this is not my field and a lot of things about accretion are rather new or complex. I can't exactly tell you this story; it is a story you must find for yourself. If you've read this far, then that's probably something you won't mind doing. And, don't forget - some of the most pretty pictures in this solar system are made by telescopes.

Update: A transition disc has been observed shortly after I first wrote this text \cite{naturetransitiondisc}. This \href{http://news.stanford.edu/pr/2015/pr-proto-planet-forming-111815.html}{press release} tells the story.

\section{The origins of life}
Life is intriguing. It seems completely different from normal matter. But how is it different? What is life? I think that is important to this story. There are a number of properties to life that are very significant. Let's just start with observations. On the scale we are familiar, we see animals, plants, birds, reptiles, amphibians, insects and fish. All of these extract sustenance of some form from the environment. All of these multiply in some way, and the resulting offspring is slightly different from the parent(s). All of these contain homoeostasis, which means that they keep their internal system different and reasonably constant compared to the environment. Note that I define homoeostasis as being active; diffusion is not sufficient.

At some point in history, we were able to fashion microscopes. In particular, the ‘ van Leeuwenhoek ’ microscope was one of the first microscopes that was relatively easy to make and could, most significantly, be taken out with the researcher. As a result, microbiology was discovered. The new forms of life discovered were far, far smaller than anything encountered before. They include the Eukaryotic micro-organisms. These are sort of living cells that have specific, membraned areas similar to an organ. These are called organelles. Furthermore, there are prokaryotic organisms which do not have these membraned organelles. Finally, there are viruses. These are not always classified as living, but they are something different from ‘ non-living ’ matter. These are a sort of parasitic life-form, which infect cells and hijack the cell for their own purposes, among which reproduction. In a way, these viruses fit all the requirements we had for larger life, but only after they have infected a cell. Before that, they are replicating molecules in a coat of some sort.

As for where life originated, I wrote a different article for that. You can find the origins of life at \weblink{share}{Share}. We'll now talk about where life, as we see it, came from. A very important molecule is Deoxyribonucleic acid, or DNA. It was discovered in the period 1869 through 1952, when it was finally confirmed that DNA was responsible for hereditary traits. Another interesting molecule is a prion. A prion is a large protein in a ‘ misfolded ’ form that acts as an infectious agent. Basically, it encounters other proteins and misfolds them as well, in largely the same way the prion itself is misfolded. It has every characteristic we noted before, except for homoeostasis.

All of the above exist. Now, we must tell the story of life, and see were these things fit in. I'm going to go for the perspective of history. As far as I know, the very first mention of a theory of descent was by Anaximander. He proposed that the first animals lived in water, when the earth was very wet; that the first land-dwelling ancestors of mankind were born in water, and spent only part of their life on land. He also argued that the very first human must've been the child of another animal, simply because human babies need prolonged nursing. A very, very keen insight, especially so when the year is about 580 BCE.

On the other hand, most humans weren't philosophers, exactly. In the eyes of most, especially those that didn't travel, one kind of animal gave birth to the same kind; and that's it. There was no bookkeeping of such things, and animal breeding hadn't been invented yet. You bred one animal with another animal of the same kind, and what you got was a new animal of that very same kind. Easy enough.

\image{dogs}{fig:dogs}{Various dogs.}
In 1809, someone called Jean-Baptiste Lamarck proposed a theory of descent. He had compared several living species to fossil forms. What he found is that age and likeliness went together. A young fossil would be very similar to a currently living species. A slightly older fossil would be similar to the first fossil, but more different from the current living species. And so on and so forth.

In 1859, Charles Darwin published ‘ On the Origin of species ’, arguably one of the most influential and well-written book ever written. I say that, because it has changed the world. The book was readable for a non-academic audience. This, in and of itself, is extremely rare. Books today on science are generally written either for an expert audience, or for a layman audience, as this article is. Good examples of the latter include, say, Bill Nye's undeniable. A delightful read in and of itself, on a far more personal level than the Origin, but written for that purpose. No, the origin was readable by just about anyone. It was still scientific, though. Charles Darwin was in the habit of not only referring to evidence, but also providing the name of the expert and being rather succinct in discribing bodies of fact.

Charles Darwin lived in a time when breeding, connected to the idea of a ‘ purebred ’, was at large. Of course, various humans of different coloration or physical size were around as well; he himself, after all, went on an extensive expedition. Anyway, Darwin lived in the age when from a common mutt, various sorts of so-called purebreds were bred and these had become visible. Consider the dogs (fig.~\ref{fig:dogs}) or their superior, cats  (fig.~\ref{fig:cats}). 

\image{cats}{fig:cats}{Various cats.}

All those different looking animals have been bred by humans. As a final example, consider this image of two bulls (fig.~\ref{fig:bulls}). To the left, you see a Belgian blue bull. The thing is ripped. It's the body builder of bulls. It's also not going to the gym every day. That, folks, is one of the most visually compelling images of biovariation I can conjure up. That's the result of a single, weird bull that some day was born into a breeder's stock. The breeder was a smart breeder, and he made sure he kept track of all its offspring. And then he crossbred the offspring of this bull again, until he had a new breed. We've got a breed of bodybuilder bulls around, due to variation.


Darwin went on a long expedition. During that, he observed a number of birds on a number of islands. These birds looked alike, but were all very different. Years after he had come back, he wrote the origin. The major insight he had is that not only does the offspring vary, but that variation was hereditary. Suppose you have some animal that's slightly different coloured. The chance that it offspring is also differently coloured is higher than for random cats. This is merely the insight basic to breeding.
\image{bulls}{fig:bulls}{Two bulls; spot the difference.}

But environments, in the sense of the land and the sense of other species, change. Within a certain species, everyone competes for the same resources. Say, nuts. Well, those birds that have some variation that allows them to be better at finding nuts, or better at reproducing, would tend to be more numerous. In a select environment, only a finite amount of resources is available, so over time the better ones are the only ones that survive. The differences can be minute; They accumulate quickly. Often, a population has some variation in all directions. When circumstances change, a larger amount of \emph{fit} individuals is present and quickly multiply.

He called this natural selection. Further along in the book, he evaluates hybrids; the offspring of distinct species. There, he describes various ways in which different species would be unable to interbreed. Not because of divine imperative, but simply because of incompatibility. For instance, because the chemical environment in the uterus of the mother would not be appropriate for the developing embryo. However, there have been various sorts of hybrid offspring that are actually fertile. This was a surprise to me as well. In my biology class, the definition for a species was that they could not interbreed and generate fertile offspring. Well, it's a nice definition but not very accurate. Consider Tigers and Lions. Two very different species of cat; both are large cats, but one is social, the other is solo. One lives on the savannah, the other in denser vegetation. A tiger is quite a bit larger than a lion.

Tigers and lions can interbreed. A tigress with a male lion gives rise a \href{https://en.wikipedia.org/wiki/Liger}{Liger}. Ligers are fertile. In 2012, Kiara, a Li-Liger was born. This would be a Liger mother and a lion father. Similarly, a Tigon is the offspring of a lioness with a male tiger. Again, it is fertile; a Li-Tigon was once born in India. The same female Tigon birthed seven more Li-Tigons. Ligers are actually larger than tigers or lions (fig.~\ref{fig:ligers}).

\image{ligers.png}{fig:ligers}{A male Lion, a Tiger and a Liger.}

Species cannot interbreed because they are not compatible. They might have issues with effective hanky-panky. The gametes might not be compatible. The mother's uterus might not be chemically suited for the hybrid offspring. There might even be some sort of rhesus-disease, where basically the mother generates antibodies that kill or damage the embryo.

Species are distinct, but only in the minds of humans. In reality, far more similarity is going on than just from outward appearances. Whales are much the same as you and I; they are mammals. In fact, their classification is Cetacea. The most similar animal living is a hippopotamus. I'm serious, these two are close together and fossils have been found that make a very good case for a common ancestor.

So, how about some evidence, then? Let me first reiterate a story Bill Nye told. You've probably heard the fish-grow-legs story. If so, then you expect that there were these gradual changes; of each stage, with some luck, there might be fossils found. The timeline was predicted; a time interval that estimated when they should have lived. At some point, scientists heard of a swamp with fossils of that age. They went there; they searched. And they found the Tiktaalik fossils; a fishlike fossil somewhere in between lobe-finned fish and four-legged animals.

Well, maybe it was some long-extinct salamander fish that just happened to be in the environment predicted at the time interval predicted. Who knows. Consider a horse. A similar fossil can be found. In fact, a full line from there to basically the first odd-toed mammal can be found. Similarly, for cows, you can keep finding these similar fossils all the way back to an even-toed mammal.

In fact, it won't surprise you to learn that most species today are related. You find distinct groups of animals; say, the horses/zebras, the cats, the wolves, the crows/ravens/magpies, and so on and so forth. And from there, you can go back a step; just as Darwin predicted, you can find a very similar fossil for an earlier stage. And from there, you can find another. And so on and so forth.

But from the odd-toed mammal and back, you can't find a fossil! So? Fossils aren't exactly rare. There are very specific circumstances required for a fossil to form. In fact, it's one of the reasons we know that the age of the earth is somewhat more than ten thousand years. A ten thousand year old earth simply would not be consistent with the amount of fossils we find. It is because of this large, large time line, that evolution can be the driving force. Similarly, it is the reason we can find so many fossils, even though fossils themselves are quite rare. There are species that have been around for periods of tens of millions of years, that have only left a few fossils for us to find. The notion of ‘ gaps in the fossil record ’ is a creationist invention. It's not actually a concern. Evolution has sufficient evidence even without the fossils. Let's get on with another piece.

Darwin based his theory on the simple realisation of biovariation and hereditary variation. Earlier, I mentioned DNA. DNA is a very large molecule that's used as information storage. It is, in a very simple sense, just a long, long chain of a few basic molecular blocks. It is traversed by other proteins in your cells, which are the agents that do stuff. The important thing here is that DNA is where hereditary information is stored. It's the reason that there is heritage of traits. And, the reason for biovariation; the copying process is not perfect, sometimes it gets damaged in a subtle way such that one block becomes another type of block, and so on. Oh, and sometimes your body messes up and parts of two different chains get switched.

It is also a prime example of evidence for common descent. All living forms on earth basically have DNA. Some have RNA and some are viruses, but the prime examples of life share DNA. The similarity in the information of DNA is not only large, but it also is more evidence for common descent. You see, forms that were taught to be similar, say, the large cats, show a larger genetic similarity to each other than a more distinct form, say, a simian. And mammals as a whole are more similar to each other than reptiles. And so on and so forth.

Finally, we must ask were life started. Many different lines of evidence are now available for the spontaneous formation of large complex molecules. Say, long chains of hydrocarbon or even aminoacids. They've been found on comets; even the recent Philae lander on 67P/Churyumov–Gerasimenko has apparently found some complex molecules on its surface.

I also mentioned Prions; a complex molecule that encounters other complex molecules and shapes them to itself, with some variation. The basic idea is simple. As more complex molecules, and different types, became available, at some point something called a replicator formed. This replicator took other molecules in its environment and made a new, slightly different, version of itself. At that point, the ingredients for evolution by natural selection are in place. Over the course of 3.7 billion years, these replicators assembled into yet larger replicators, at some point into a sort of proto-cell. Basically, it's just a small membraned environment which it could regulate, apparently so it could take more time to assemble new copies from molecules taken from its environment. From cells, we get to multicellular organisms. It has been shown several times over that this is not an exciting transition; single celled yeast in a centrifuge, for instance, rapidly becomes multicellular. From multi-cellular organisms, we get to micro organisms; from there, we move upwards to arrive at small, more creature-like organisms. And these assembled into yet larger systems; ultimately leading to the familiar forms.

The replicators haven't been found. I do not think it is likely that we will find evidence of this proto-life, it being just a set of molecules that probably got assembled into the larger forms. Instead, I do think it likely that at one point we can make these replicators. Prions are, to me, evidence that replicators could be the answer. The entire process is called abiogenesis; the natural, spontaneous emergence of life.

\section{Conclusion}
I have attempted to tell you of the scientific method. I've outlined the story of life, the universe and everything. Here and there, I have attempted to illustrate some of the evidence for these theories or hypotheses. But don't take it from me. Go out and search for this stuff. Those of you at a university are in an excellent position, as most universities have access to all the journals in relevant fields. Read other books. The science of Discworld series is a very nice example of what I tried to do. A series of multiple books by the author Terry Pratchett together with Ian Stewart and Jack Cohen, it tells the same story I just told you in far greater detail and with a lot more humour. Or opt for something like Bill Nye's recent publication, Undeniable. Or, if you're up for it, Richard Dawkins selfish gene or the God Delusion. Of course, you can also opt for all sorts of other popular science publications. These are the ones that spring to mind.
\end{multicols}