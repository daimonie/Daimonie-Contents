\abstract{This is the home page. The home page explains the name of this website. It also explains its purpose and present the outline. Finally, it introduces the traits of this website.}


\begin{multicols}{2}

\section{Welcome!} 


To \emph{Daimonie.com}, a website accompanying \href{https://www.facebook.com/daimonie}{my Facebook Page} and \href{./blog"}{my Blog}.

I am a graduate of Delft university of Technology in the Netherlands. I studied Physics, specialising towards Quantum condensed matter theory. Previously, I was employed at the Nynke Dekker Lab at Delft University of Technology as a technician, writing analysis and setup control software. I've since then moved on to the Lorentz Institute for Theoretical Physics in Leiden, where I am engaged as a PhD student working on numerical AdS/CFT for strange metals. 

My pseudonym is Daimonie, \href{https://goo.gl/xwLXc1}{an ancient Greek} word that I first saw used in  the context of "extraordinary one", in Homer's Illiad.  It was used by Helen of Troy to address a magically disguised Aphrodite, goddess of love. The word is also part of the word for \href{https://goo.gl/kktWCr}{Spartan}, and has \href{https://goo.gl/sKxvMo}{Demon as a derived term}. As you can see, it is an interesting word, even if I did first translate it as 'weirdo'.

I have noted that there is some confusion among readers regarding my current status. I have a M. Sc. degree in Physics, so that I should be allowed to refer to myself as a physicist. I am currently employed at a university. I understand that I have an international audience, so let me shortly detail education in the Netherlands. In high school, we have multiple levels, one of which prepares for an academic tertiary education. Academic tertiary education starts with a Bachelor of three years followed by a Master of two years. After the Master, one can pursue a PhD degree, which is a paid position. The Master of Science degree is a postgraduate degree.


\section{Outline}  
Here, I explain the other pages of my website. You are currently on the home page.

The first page, \weblink{origins}{Origins} is a long story of the known history of the universe. It is the story of origins, explaining how we got from the Big Bang to human life and ingenuity.

The second page, \weblink{sceptic}{Sceptic} explains one of my common time-sinks: discussion with conspiracy theorists and science deniers. They're not exactly the same, but conversation with them often feels like that. It also explains why this is one of my time-sinks.

The third page, \weblink{science}{Science} describes the scientific method and attitude. It also elaborates on the difference between Historical and Observational science.

The fourth page, \weblink{share}{Share} is a repository of files I share to the web. This includes PDF versions of each page, but also attachments, short blog articles and resources used in blogs. 

The fifth page, \weblink{blog}{Blog} is a gateway to my blog. It contains a short preview list of each article, including a short description. 

\section{User Interface}

This website has a number of peculiar traits. For instance, I built a system to include citations. For example, the Credible Hulk \cite{crediblehulk}. Clicking on the citation will open a panel at the bottom showing the citation information. This generally includes a link.

You can open the side navigation by pressing the Menu icon on the top left. On desktops and other large screens, it is locked open. In this menu, you can find a table of contents for the current page. Additionally, on very small screens the page navigation can eb found here.

On some pages, I refer to specific figures. Usually, this looks similar to a citation but includes a \emph{fig:} label. Clicking on these brings you to the figure. At the bottom of the figure, there is a link that brings you back to the reference.
\end{multicols}
