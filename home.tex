\begin{abstract}
    This is the default page. It describes the name and purpose of this website.
\end{abstract}


\begin{multicols}{2}

\section{Welcome!} 


To \emph{Daimonie.com}, a website accompanying \href{https://www.facebook.com/daimonie}{my Facebook Page} and
\href{./blog"}{my Blog}. My pseudonym is Daimonie,
\href{https://goo.gl/xwLXc1}{an ancient Greek} word
that I first saw used in  the context of "extraordinary one", in Homer's Illiad.  


It was used by Helen of Troy to address a magically disguised Aphrodite, goddess of love. The word is also part of the word for
\href{https://goo.gl/kktWCr}{Spartan}, and has \href{https://goo.gl/sKxvMo}{Demon as a derived term}. As you can see, it is an interesting word, even if I did first translate it as 'weirdo'.

I am a graduate of Delft University of Technology, having completed the Master of Science programme in Physics, specialising towards
Quantum Condensed Matter Physics. I am also a somewhat accomplished programmer and a lover of science in general. I also fervently believe
in the value of widespread scientific literacy, resulting in some science activism. 

I have noted that there is some confusion among readers regarding my current status. I have a M. Sc. degree in Physics, so that I should be
allowed to refer to myself as a physicist. I will likely pursue a PhD in theoretical quantum condensed matter, but have not done so yet. I understand
that I have an international audience, so let me shortly detail education in the Netherlands. In high school, we have multiple levels of which
one prepares for an academic tertiary education. Academic tertiary education starts with a Bachelor of three years followed by a Master.
After the Master, one can pursue a PhD degree, which is a paid position. The Master of Science degree is a postgraduate degree.


\section{	Contents } 


On this website, you will find several pages. The first, titled \emph{ Origins }, is a very large article telling the story of origins.
That is, it tells you of the first instants of the observable universe, how stars and planets form, and how life emerged and adapted. The
second page, titled \emph{Share}, contains a list of items shared on my website. These are generally stories in printable formats that I
have written. The third button leads to my blog. The fourth, titled \emph{Scepticism}, tells you about what I mean by that term and gives
some examples. Finally, the fifth page is titled \emph{Science}, where I explain the scientific method and the critical attitude accompanying
it.

Finally, I wanted to point out how \emph{citations} work. After I make a claim that I think needs a citation, you will find a little button
that shows the full citation. For instance, I am third author on a paper on generalised nematics. You should see an icon to the right;
you can hover or click it. By doing so, the full reference should become visible. Hover out or click again to hide the reference. Sometimes, you
see a link rather than a reference; this is probably because of the nature of the source.

No doubt you are now either wondering why someone would go to the lengths to cite on his webpage, or you really want to read a story that has
citations. In the first case, I suggest you visit the \href{./science}{Science} section, otherwise you might be interested in reading
\href{./origins}{the origins story}.
\end{multicols}