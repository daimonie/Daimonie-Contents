\abstract{
In this article, I describe the scientific method and the attitude that accompanies it. The goal is to provide a baseline for the other articles on my website and blog. This is far from trivial, as science communicators encounter many people that seem to have no introduction in either.}
\begin{multicols}{2}
\section{Science}
A lack of familiarity with the scientific method and attitude can be problematic. We live in a society that continuously increases the amount of information available and assailing each person. As a result, the scientific attitude is almost a requirement to sift fact from fiction, news from false news and good advice from alternative fact. 

As an example, I discuss a common creationist tactic. The tactic is to paint a difference between observational and historical science. Is there a real difference? Many laymen cannot assess the answer, even though it should be extremely plain. At the end of the article, we should be able to do so.

\section{Scientific thought}

The origins of the scientific method trace its way throughout history. Precursors have been found as far back as 1600 BCE, after which we encounter a number of Greek and Arab philosophers (\href{http://en.wikipedia.org/wiki/List_of_ancient_Greek_philosophers}{Wiki}). For instance, Ibn al-Haytham already used experimentation to find the results of his Book of Optics. He was one of the better inspirations for the emergence of the emperical method.

I think one of the most significant contributions was made by Galileo Galileo (\href{https://en.wikipedia.org/wiki/Galileo_Galilei}{Wiki}). The reason is that Galileo gave evidence precedence. The anecdote is quite accessible and the principles involved feature in a Mythbusters fragment.  

Imagine or google a picture of a galleon (\href{https://en.wikipedia.org/wiki/Galleon}{Wiki}), a multi-decked sailing ship. You are at the top of the centre mast, looking down the dazzling distance. For science, you are about to drop a waterballoon filled with red paint. Before you do so, you wonder where it will land.

Three scenarios float through your mind. First, there is no wind, so without the ship moving you expect it to go straight down. If the ship moves, however, then two scenarios might reflect reality. You drop the balloon, but the ship is moving. So, the balloon comes down behind the base of the mast. However, is the balloon not moving with the ship before you drop it? The third scenario is that it comes down at the base of the mast.

This question was discussed by various people, and the answer was accepted at the time: It would come down behind the mast. The reason was of course that the ship moved. Many famous people claimed that other famous people had done the experiment, or at least someone they knew via a cousin,	twice removed, whom knew a guy whose niece dated someone that did. Either way, Galileo doubted the orthodox answer and went out and did the experiment. I don't know where he found the galleon. 

Naturally, the point of impact was at the base of the mast. The balloon, after all, was moving with the ship and no effort was made to stop it. This demonstrated a principle now known as Galilean relativity. If we have a camera centred on the balloon, moving with it, then the balloon seems at rest and then falls straight down as expected because of gravity. If we have a camera centred on a point outside the ship, we see the balloon moving sideways and down. The difference being the movement of the focal point of the camera. Both frames of references are correct, and they are connected by relativity. In the Mythbusters fragment, they shoot a ball out of the back of a moving vehicle. If you centre the camera on the car, you see that the ball moves backwards at some velocity. They matched this velocity to the constant forward velocity of the vehicle. As a result, a camera centred on the Horizon sees the ball going backwards and forward with that constant speed - that is, it does not see it move. The ball drops straight down in a beautiful demonstration of classical physics.

The principle Galileo demonstrated is that empirical data takes precedence over theory and established credentials. More recently, the Mythbusters attempted to teach this principle to everyone, especially children. Experimental results rule supreme. It is this thought that truly meant, in my opinion, the rise of the scientific method. While others contributed, greatly, Galileo demonstrated the philosophical principles.

\section{Scientific Method}

The Scientific Method describes a method of acquiring new knowledge based on empirical or measurable evidence. It is inductive, a precise term describing the way conclusions are reached. In its antithesis, deductive reasoning, if the premises are true it follows that the conclusion is also true. The conclusions follow logically from the premises. For instance, the conclusion [B is an ape] follows logically from the premise [all men are apes] and the premise [B is a man]. If the premises are satisfied, which in this case they are, the conclusion must be true.

In inductive reasoning, the conclusion is not sure. If the premises are satisfied, the conclusion is credible. For instance, at some point in history all known swans were white. Therefore, if we discover a new swan it will [probably] be white. In this instance, we found black swans in Australia. A far stronger example is that, since all rocks fall, if we pick up an arbitrary rock will also fall. This sounds ludicrous, because the conclusion is so very credible. Even so, it is a conclusion reached by inductive reasoning.

Why do rocks fall? By now, everyone knows the answer by rote: Gravity. How does that work? Sir Isaac Newton hypothesised a specific formula that describes the gravitational attraction between two objects with mass. Since then, we have amassed a body of evidence. Given the body of evidence, which are measurements and in this case satisfied premises, we conclude that Newton's hypothesis is very credible or likely. For a few hundred years, no new evidence was found that contradicted Newton's law of gravitation. The word \emph{Law} emphasises that the formula is merely what you can find in the data, and does not describe the explanation.

Einstein came along with his General Relativity, which is his theory of gravity (among other things). New evidence found contradicted Newton's law of gravity. In this case, the amount of evidence increased so that new premises were added that are incompatible with the conclusion. As a result, Newton's law become less credible. Today, it has lost credibility; scientists consider it falsified. Einstein's theory is so far the most credible hypothesis given the evidence. Einstein's theory is an explanation, not a mere data trend. This is why it is not called a Law. 

The \emph{scientific method} starts out by taking the existing body of evidence and making a hypothesis, which are the inductive premises and conclusion. The hypothesis is tested by experiment, adding more evidence or premises. This can either increase or lower the credibility of the conclusion. In some cases where the evidence clearly invalidates the hypothesis it loses al credibility and this is called falsification. In a sense, it never ends; we can always increase the body of evidence or amount of premises.

The pedestrian scientific method, the simplified version taught in high schools and all over the internet, can be stated as a simple list:
 
\itememph{Assess}{ the existing body of evidence.}
\itememph{Hypothesise}{ a formula that fits the body of evidence. }
\itememph{Experiment}{: adding new observational evidence that might confirm or invalidate the hypothesis. }
\itememph{Peer-review}{: Ask your peers to review your work or repeat your experiments. }
\itememph{Repeat}{, ad infinitum et ultra.} 

You might have noticed that I mentioned a formula explicitly. Formula are important; they tell you about fixed behaviours in Nature. It does not necessarily imply mathematical formula, although it often does. Formula that become ridiculously credible are often called Laws, e.g. Newton's law of gravity, the ideal gas law or Hooke's law.

But where is the \emph{explanation}? Explanations are descriptions of Nature that lead to formula. In this sense, they are hypothesis-generating devices. An Explanation that has led to very credible hypotheses or laws is often called a \emph{scientific theory}.

Sometimes, two theories are both credible. How do you know which one describes Nature? This is determined by a \emph{crucial experiment}, where two theories are tested in circumstances where their hypotheses conflict. As a good example of crucial evidence, the gravitational lensing effect was predicted by both Newton's and Einstein's theories of gravity. Gravitational lensing means light bending around a massive object due to gravity. The hypotheses the two theories generated conflicted; the amount of bending predicted by Newton was only half of that predicted by Einstein. In 1919, Arthur Eddington, Frank Watson Dyson and their collaborators performed the measurements during a solar eclipse and they confirmed the hypothesis of Einstein's theory but falsified that of Newton's theory. In other words, Einstein was correct.

\section{Observational and Historical science}

As mentioned earlier, creationists often try to paint a difference between observational and historical science. They claim that observational science is real science, and historical science is just anti-scripture speculation. For example, Ken Ham attempted to do this in his debate with Bill Nye the Science Guy.

Let's start with a simple example. The theories of radioactive matter, long made credible in nuclear and radiation research, predicted a method now known as radiometric dating. The carbon example is simplest; in our atmosphere, carbon-14 is spontaneously formed in our atmosphere, and both carbon-12 and carbon-14 are absorbed by carbon-based lifeforms. At the moment an organism dies, it has a specific ratio of carbon-14 and carbon-12 in its body which reflects the ratio in the atmosphere (during its lifetime). However, carbon-14 is slightly radioactive. A given number of carbon-14 atoms will halve in about 5730 years, the decaying half turning into nitrogen-14. 

The hypothesis is that the earth is a certain age. At first, this age was relatively young; about 6000 for a young-earth creationist. If the earth is that old, then radiometric dating methods allow us to measure rocks to see their age. The hypothesis of a 6000-year old earth was instantly falsified because most rocks are older. A lot of organic fossils are too; carbon-14 dating allows you to date objects younger than about 70 thousand years.

A new hypothesis was formed, based on the ages of the samples used to falsify the previous hypothesis. This hypothesis was confirmed, again and again, leading to the current estimate of 4.543 billion years. Creationists claim that this is historical science, which is speculation. But you have just seen that the scientific method was applied just as it is elsewhere; form a hypothesis, add new evidence, see if it confirms or falsifies, repeat. 

There are more examples, but there is no merit in discussing them all. A large number of scientific theories deal with matters that do not fit the creationist world view, and have been deemed historical science by them. However, there is no such difference; the knowledge is acquired by the same method. The big difference is that creationists do not like "historical science" conclusions, and dismiss them out of hand.

An often heard argument is that scientists assume that the natural laws did not change during that period. This is true; that is one of the many base hypotheses of science. However, it is also one of the oldest theories; it has been confirmed again and again. We can take the creationist explanation that the natural laws did change, and find that experiment falsifies this. Another is that we should consider that their deity has made this world in such a way that it confuses us. Well, dragons might have the ability to turn invisible. What can be asserted without evidence can be dismissed without evidence. 
\end{multicols}